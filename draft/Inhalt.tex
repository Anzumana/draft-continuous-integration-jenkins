% Hier k�nnen die einzelnen Kapitel inkludiert werden. 
% Die Dateien m�ssen auf .tex enden. Diese Endung muss
% beim Inkludieren aber weggelassen werden.
% Info: \include und \input unterscheiden sich im wesentlichen darin, dass bei
% \include immer eine neue Seite angefangen wird.

\chapter{Einleitung}
\label{cha:Einleitung} % Ein Label ist optional, erm�glicht aber die Referenzierung
Flie�textFlie�textFlie�textFlie�text
\section{Motivation}
\label{sec:Motivation} % Ein Label ist optional, erm�glicht aber die Referenzierung 
Flie�textFlie�textFlie�textFlie�text
\section{Ziel der Arbeit}
Flie�textFlie�textFlie�textFlie�text
\section{Aufbau der Arbeit}
Flie�text Flie�text Flie�text Flie�text Flie�text Flie�text Flie�text Flie�text
Flie�text Flie�text Flie�text Flie�text Flie�text Flie�text Flie�text Flie�text
Flie�text Flie�text Flie�text Flie�text Flie�text Flie�text Flie�text Flie�text
Flie�text Flie�text Flie�text Flie�text Flie�text Flie�text Flie�text Flie�text


\chapter{Ebenen}
In diesem Kapitel wird kurz aufgezeigt,
welche Gliederungsebenen es gibt. Hier befinden wir uns auf der Ebene eines
\fett{Kapitels}. %\fett ist ist selbst-erstelltes Kommando, dass den gleichen Effekt wie \textbf hat
\section{Abschnitt}
\subsection{Unterabschnitt}
\subsubsection{Unter-Unterabschnitt}
Soll noch tiefer gegliedert werden ist eine �nderung der Formatierungen
und/oder das Verwenden eines zus�tzlichen Paketes notwendig.

\include{Inhalt/scrivener/copy/01}
\include{Inhalt/scrivener/copy/02}
\include{Inhalt/scrivener/copy/03}
\include{Inhalt/scrivener/copy/04}
\include{Inhalt/scrivener/copy/05}
\include{Inhalt/scrivener/copy/06}
\include{Inhalt/scrivener/copy/07}
\include{Inhalt/scrivener/copy/08}
\include{Inhalt/scrivener/copy/09}
\include{Inhalt/scrivener/copy/10}
\include{Inhalt/scrivener/copy/11}
\include{Inhalt/scrivener/copy/12}
\include{Inhalt/scrivener/copy/13}
\include{Inhalt/scrivener/copy/14}
\include{Inhalt/scrivener/copy/15}
\include{Inhalt/scrivener/copy/16}
\include{Inhalt/scrivener/copy/17}
\include{Inhalt/scrivener/copy/18}
\include{Inhalt/scrivener/copy/19}
\chapter{Schlussbemerkungen}
\label{cha:Schlussbemerkungen} % Ein Label ist optional, erm�glicht aber die Referenzierung
Flie�textFlie�textFlie�textFlie�text
\section{Fazit}
Flie�textFlie�textFlie�textFlie�text
\section{Ausblick}
Flie�textFlie�textFlie�textFlie�text

