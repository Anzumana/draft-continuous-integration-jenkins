% Hier k�nnen die einzelnen Kapitel inkludiert werden. 
% Die Dateien m�ssen auf .tex enden. Diese Endung muss
% beim Inkludieren aber weggelassen werden.
% Info: \include und \input unterscheiden sich im wesentlichen darin, dass bei
% \include immer eine neue Seite angefangen wird.

\include{Inhalt/Einleitung}
\include{Inhalt/Ebenen}
\include{Inhalt/scrivener/copy/01}
\include{Inhalt/scrivener/copy/02}
\include{Inhalt/scrivener/copy/03}
\include{Inhalt/scrivener/copy/04}
\include{Inhalt/scrivener/copy/05}
\include{Inhalt/scrivener/copy/06}
\include{Inhalt/scrivener/copy/07}
\include{Inhalt/scrivener/copy/08}
\include{Inhalt/scrivener/copy/09}
\include{Inhalt/scrivener/copy/10}
\include{Inhalt/scrivener/copy/11}
\include{Inhalt/scrivener/copy/12}
\include{Inhalt/scrivener/copy/13}
\include{Inhalt/scrivener/copy/14}
\include{Inhalt/scrivener/copy/15}
\include{Inhalt/scrivener/copy/16}
\include{Inhalt/scrivener/copy/17}
\include{Inhalt/scrivener/copy/18}
\include{Inhalt/scrivener/copy/19}
\include{Inhalt/Fazit}
